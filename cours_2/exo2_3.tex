\documentclass{article}
\usepackage{fontspec}
\usepackage{xunicode}
\usepackage{polyglossia}
\setmainlanguage{french}



\begin{document}

\section{csquotes}
% mettre toute la citation de VH entre guillemets, et, au sein de cette citation, celle de Napoléon également.
% ils = les hommes de génie

Victor Hugo a écrit dans la Préface de Cromwell: C’est par là qu’ils  touchent à l’humanité, c’est par là qu’ils sont dramatiques. Du sublime au ridicule il n’y a qu’un pas, disait Napoléon, quand il fut convaincu d’être homme.

\section{quote}

%  Citation un peu plus longue, d'un sectione, à mettre dans l'environnement adapté.
% Mettre des points de suspension à la fin entre crochets, suivis d'un commentaire entre crochets, et d'une note de bas de page indiquant: Dans la Préface de Ruy Blas.

Victor Hugo présente ainsi le public d'une pièce de théâtre:

Trois espèces de spectateurs composent ce qu’on est convenu d’appeler le public: premièrement, les femmes; deuxièmement, les penseurs; troisièmement, la foule proprement dite. Ce que la foule demande presque exclusivement à l’œuvre dramatique, c’est de l’action ; ce que les femmes y veulent avant tout, c’est de la passion ; ce qu’y cherchent plus spécialement les penseurs, ce sont des caractères.


\section{quotation}
% Citation sur plusieurs sectiones, à mettre dans l'environnement adapté

Homère, en effet, domine la société antique. Dans cette société, tout est simple, tout est épique. La poésie est religion, la religion est loi. À la virginité du premier âge a succédé la chasteté du second. Une sorte de gravité solennelle s’est empreinte partout, dans les mœurs domestiques comme dans les mœurs publiques. Les peuples n’ont conservé de la vie errante que le respect de l’étranger et du voyageur. La famille a une patrie ; tout l’y attache ; il y a le culte du foyer, le culte des tombeaux.

Nous le répétons, l’expression d’une pareille civilisation ne peut être que l’épopée. L’épopée y prendra plusieurs formes, mais ne perdra jamais son caractère. Pindare est plus sacerdotal que patriarchal, plus épique que lyrique. Si les annalistes, contemporains nécessaires de ce second âge du monde, se mettent à recueillir les traditions et commencent à compter avec les siècles, ils ont beau faire, la chronologie ne peut chasser la poésie ; l’histoire reste épopée. Hérode est un Homère.

(Victor Hugo, Cromwell, Préface)

\section{url}
% mettre un hyperlien
Pour avoir le texte complet, suivez le lien %  https://fr.wikisource.org/wiki/Cromwell_-_Préface


\section{verse}

Amusons-nous un peu avec Edward \textsc{Lear}, \emph{The owl and the pussy cat, and other nonsens poetry}
%Mettre en forme le poème suivant. Les vers 3 et 4 sont indentés

There was an old man who when little
Fell casually into a kettle
	But, growing too stout,
	He could never get out,
So he passed all his life in that kettle


\end{document}
