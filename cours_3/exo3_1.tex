\documentclass[12pt]{article}
\usepackage{fontspec}
\usepackage{xunicode}
\usepackage{polyglossia}
\setmainlanguage{french}
\usepackage{csquotes}


\usepackage{graphicx}%insérer des images
\usepackage{adjustbox}%ajuster facilement la taille des images
\usepackage{lscape}%mettre en format paysage
\usepackage{tikz}%faire des graphiques

\usepackage{pgfplots}%graphiques à partir de données (courbes, histogrammes)
\usepackage{pgfplotstable}%importer les données depuis un fichier csv
%\pgfplotsset{/pgf/number format/1000 sep=, /pgf/number format/read comma as period}%Ne pas mettre de virgule pour les nombres au-dessus de mille; interpréter la virgule comme élément d'un nombre décimal
%\pgfplotstableset{%
%	assign column name/.style={/pgfplots/table/column name={\textbf{#1}}},%quand on importe un tableau, le titre des colonnes apparaît en gras
%	column type=l}%et les colonnes sont alignées à gauche

\begin{document}
	
\section{Insérer des flottants}
\subsection{Images}

%1/Insérez ici l'image genealogie.jpg graĉe à \includegraphics (RQ: Source de l'image: https://books.openedition.org/efr/2282)
%2/Modifiez l'échelle de l'image pour que ce soit satisfaisant
%3/testez les modifications d'échelle en remplaçant par \adjincludegraphics 
%4/ Transformez l'image en flottant avec l'environnement figure. Indiquez où positionner le flottant.
%5/Ajoutez-lui une légende, ainsi que la source, qui se trouve ici: https://books.openedition.org/efr/2282. La
%6/ Enlever les modifications d'échelle et faites-la tourner au format paysage
%7/ A la fin du document, insérez une liste des figures




\subsection{Tableaux}

%1/en utilisant l'outil 'Assistant - Tableau ' de Texstudio: Insérez ici un tableau de trois colonnes et trois lignes; la première ligne des deux premières colonnes doit être fusionnée
%2/Observez comment le tableau est construit, puis rajoutez une ligne
%3/ Faites un autre tableau en choisissant pour les colonnes, dans l'option 'alignement', une des possibilités de 'largeur fixée'. Modifiez ensuite la largeur
%4/Transformer vos tableaux en flottant grâce à l'environnement table, et insérez une légende.
%5/ imprimez à la fin du document la liste des tableaux


\subsection{Faire des graphiques avec tikz}

\subsubsection{Comprendre la syntaxe de tikz: l'exemple du \emph{stemma codicum}}

% 1/Décommentez le stemma, compilez et observez
% Ajoutezajoutez :
% a) un enfant à A, appelé D
% b) 1 enfant E à D
% Faites bien attention aux accolades. Indentez le code pour y voir plus clair
%mettez le tout dans un flottant, indiquez un paramètre de placement et ajoutez une légende

\begin{tikzpicture}
%\node {A}
%	child {node {B}}
%	child {node {C}}
%	;
\end{tikzpicture}


\subsubsection{Tracer des diagrammes avec tikz  et pgfplot}

% Faire un schéma à partir du tableau 'travail_domestique_insee.csv' 

\begin{tikzpicture}
	\begin{axis}%[xtick=data]
		\addplot coordinates {};
		\addplot coordinates {};
		\legend{}
	\end{axis}
\end{tikzpicture}

\section{Importer des données avec pgfplotstable pour produire tables et graphiques}

%1/Importez le tableau travail_domestique_insee.csv

%\pgfplotstableread[col sep=comma]{}{}

%\pgfplotstabletypesetfile{}



%2/importez les données de ce tableau sous forme de diagramme
\begin{tikzpicture}
%	\begin{axis}%[xtick=data]
%		\addplot table[]{};
%		\addplot table[]{};
%	\end{axis}
\end{tikzpicture}

%3/Importez le tableau classement_prenoms_insee.csv sous forme d'histogramme en barres
%\pgfplotstableread[col sep=comma]{}{}

\begin{tikzpicture}[xscale=1.5]
%	\begin{axis}%[ybar, xticklabels from table={}{}, xtick=data, x tick label style={rotate=90}]
%		\addplot table[]{};
%	\end{axis}
\end{tikzpicture}


%Importez le tableau insee_categories_socio_professionnelles.ods. Attention, il faut auparavant le transformer en csv et le nettoyer.

%\pgfplotstableread[col sep=]{}{}
%\pgfplotstabletypesetfile[string type]{}



\section{Liste des figures et des tables}
%Insérer ici: une liste des figures et une liste des tables





\end{document}