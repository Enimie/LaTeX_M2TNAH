\documentclass[12pt,twoside]{book}
\usepackage{fontspec}
\usepackage{xunicode}
%\usepackage{polyglossia}
%\setmainlanguage{french}
\usepackage[french]{babel}

\usepackage{imakeidx}%pour faire un index
\makeindex[intoc]%commande pour générer l'index




 \usepackage{hyperref}%faire des liens (url, renvois internes, etc)
\usepackage[toc=true]{glossaries}%doit être appelé après hyperref (exception à la règle)
\makeglossaries





%%%% définir des entrées de glossaire
\newglossaryentry{ex}{%
	name={exemple},%
	description={illustration% 
		d'un concept}}

\newglossaryentry{co}{%
	name={concept},%
	description={Idée générale, représentation mentale et abstraite que l’on a d’un objet}}


\title{Faire des renvois internes et des index}
\author{Cours de méthodologie}

%indiquer des règles d'hyphénation pour des mots précis
\begin{hyphenrules}{french}
	\hyphenation{}
\end{hyphenrules}

\begin{document}
\maketitle

\section*{Objectifs de l'exercice}\addcontentsline{toc}{section}{Objectifs de l'exercices}
\begin{enumerate}
  \item Apprendre à faire des renvois internes grâce à des labels
% Placez un label \label{clef} dans l'environnement table, après caption, et un autre juste après le nom d'une section ou d'une subsection
% Rour renvoyer au label: \pageref{clef} imprimera la page, \ref{clef} numéro de section/de figure; \nameref{clef} le  titre de la section ou du flottant.
% Testez en décommentant la footnote à la fin du texte et en mettant dans les commande le nom du label voulu.

  \item  Faire un ou plusieurs index
% 1/Index simple
%en débutd de chaque section se trouve deux noms propres. Indexez-les par la commande \index{Nom à indexer}
% Attention: ce qui est mis dans index n'apparait PAS dans le texte. Il faut donc avoir par exemple Auguste\index{Auguste}
% \printindex en fin de document à décommenter
%  Régler le problème avec élisabeth: \index{Elisabeth@Élisabeth} 
% Rajoutez \index{Napoléon|textbf} et testez
% Ajouter Bonaparte\index{Bonaparte|see{Napoléon}}\index{Napoléon} et testez

 % 2/ Faire plusieurs index
%\makeindex[intoc, title=Index des personnes]
%\makeindex[intoc, name=loc, title=Index des noms de lieux]
% ajouter en argument optionnel à la commande index [loc] pour les noms de lieux
%après \printindex, ajouter \printindex[loc]

\item Faire un glossaire 
%Observez les entrées de glossaires définies dans le préambule
%appeler dans le corps du texte les deux mots définis dans le glossaire au moyen de la commande \gls{entree_du_glossaire}
%Compiler une première fois. Comme vous avez déjà un index existant, pour compiler le glossaire il faut lancer de le terminal makeglossaries nom_du_fichier. Texstudio le fait pour vous: outl-glossaire (F9)
% Créer une nouvelle entrée et testez-la
% Changer le titre du glossaire: en option à printglossary, title=le nouveau titre
% Pour le faire apparaître dans la table des matière: option à l'appel de package: toc=true

\end{enumerate}

\section{Première partie}

Élisabeth à Londres 

Lorem ipsum dolor sit amet, consectetuer adipiscing elit. Ut purus elit,
vestibulumut, placerat ac, adipiscing vitae, felis. Curabitur dictumgravida
mauris. Namarcu libero, nonummy eget, consectetuer id, vulputate a, magna. Donec vehicula augue eu neque. Pellentesque habitant morbi tristique
senectus et netus et malesuada fames ac turpis egestas. Mauris ut leo. Cras
viverra metus rhoncus sem. Nulla et lectus vestibulumurna fringilla ultrices.
Phasellus eu tellus sit amet tortor gravida placerat. Integer sapien est, ia-
culis in, pretiumquis, viverra ac, nunc. Praesent eget semvel leo ultrices
bibendum. Aenean faucibus. Morbi dolor nulla, malesuada eu, pulvinar at,
mollis ac, nulla. Curabitur auctor semper nulla. Donec varius orci eget risus.
Duis nibh mi, congue eu, accumsan eleifend, sagittis quis, diam. Duis eget
orci sit amet orci dignissimrutrum.


\begin{table}[h]
  \begin{center}
    \begin{tabular}{|c|c|}
      \hline
      rosa & rosae \\
      \hline
      rosa & rosae \\
      \hline
      rosam & rosas \\
      \hline
      rosae & rosarum \\
      \hline
      rosae & rosis \\
      \hline
      rosa & rosis \\
      \hline
    \end{tabular}
  \end{center}
    \caption{Une déclinaison}
  \end{table}

Nam dui ligula, fringilla a, euismod sodales, sollicitudin vel, wisi. Morbi
auctor loremnonjusto. Namlacus libero, pretiumat, lobortis vitae, ultricies
et, tellus. Donec aliquet, tortor sed accumsan bibendum, erat ligula aliquet
magna, vitae ornare odio metus a mi. Morbi ac orci et nisl hendrerit mol-
lis. Suspendisse ut massa. Cras nec ante. Pellentesque a nulla. Cumsociis
natoque penatibus et magnis dis parturient montes, nascetur ridiculus mus.
Aliquamtincidunt urna. Nulla ullamcorper vestibulumturpis. Pellentesque
cursus luctus mauris.

\section{Deuxième partie}

Charlemagne à Aix


Nulla malesuada porttitor diam. Donec felis erat, congue non, volut-
pat at, tincidunt tristique, libero. Vivamus viverra fermentumfelis. Donec
nonummy pellentesque ante. Phasellus adipiscing semper elit. Proin fermen-
tummassa ac quam. Sed diamturpis, molestie vitae, placerat a, molestie
nec, leo. Maecenas lacinia. Namipsumligula, eleifendat, accumsannec, sus-
cipit a, ipsum. Morbi blandit ligula feugiat magna. Nunc eleifend consequat
lorem. Sed lacinia nulla vitae enim. Pellentesque tincidunt purus vel magna.
Integer non enim. Praesent euismod nunc eu purus. Donec bibendumquam
in tellus. Nullamcursus pulvinar lectus. Donec et mi. Namvulputate metus
eu enim. Vestibulumpellentesque felis eu massa.

\subsection{Une sous-partie}
Quisque ullamcorper placerat ipsum. Cras nibh. Morbi vel justo vitae
lacus tincidunt ultrices. Loremipsumdolor sit amet, consectetuer adipiscing
elit. Inhac habitasse platea dictumst. Integer tempus convallis augue. Etiam
facilisis. Nunc elementumfermentumwisi. Aenean placerat. Ut imperdiet,
enimsedgravida sollicitudin, felis odio placerat quam, ac pulvinar elit purus
eget enim. Nunc vitae tortor. Prointempus nibh sit amet nisl. Vivamus quis
tortor vitae risus porta vehicula.

\subsubsection{Une sous-sous partie}
Fusce mauris. Vestibulum luctus nibh at lectus. Sed bibendum, nulla
a faucibus semper, leo velit ultricies tellus, ac venenatis arcu wisi vel nisl.
Vestibulumdiam. Aliquampellentesque, augue quis sagittis posuere, turpis
lacus congue quam, in hendrerit risus eros eget felis. Maecenas eget erat in
sapien mattis porttitor. Vestibulumporttitor. Nulla facilisi. Sed a turpis eu
lacus commodo facilisis. Morbi fringilla, wisi in ssim interdum, justo
lectus sagittis dui, et vehicula libero dui cursus dui. Mauris tempor ligula
sed lacus. Duis cursus enim ut augue. Cras ac magna. Cras nulla. Nulla
egestas. Curabitur a leo. Quisque egestas wisi eget nunc. Nam feugiat lacus
vel est. Curabitur consectetuer.

\section{Troisième partie}

Napoléon à Paris


\subsection{Encore une}

Suspendisse vel felis. Ut lorem lorem, interdum eu, tincidunt sit amet,
laoreet vitae, arcu. Aenean faucibus pede eu ante. Praesent enim elit, rutrum
at, molestie non, nonummy vel, nisl. Ut lectus eros, malesuada sit amet, fer-
mentum eu, sodales cursus, magna. Donec eu purus. Quisque vehicula, urna
sed ultricies auctor, pede lorem egestas dui, et convallis elit erat sed nulla.
Donec luctus. Curabitur et nunc. Aliquam dolor odio, commodo pretium,
ultricies non, pharetra in, velit. Integer arcu est, nonummy in, fermentum
faucibus, egestas vel, odio.

Sed commodo posuere pede. Mauris ut est. Ut quis purus. Sed ac odio.
Sed vehicula hendrerit sem. Duis non odio. Morbi ut dui. Sed accumsan ri-
sus eget odio. In hac habitasse platea dictumst. Pellentesque non elit. Fusce
sed justo eu urna porta tincidunt. Mauris felis odio, sollicitudin sed, volut-
pat a, ornare ac, erat. Morbi quis dolor. Donec pellentesque, erat ac sagittis
semper, nunc dui lobortis purus, quis congue purus metus ultricies tellus.
Proin et quam. Class aptent taciti sociosqu ad litora torquent per conu-
bia nostra, per inceptos hymenaeos. Praesent sapien turpis, fermentum vel,
eleifend faucibus, vehicula eu, lacus.

\subsection{La dernière}
Pellentesque habitant morbi tristique senectus et netus et malesuada
fames ac turpis egestas. Donec odio elit, dictum in, hendrerit sit amet, eges-
tas sed, leo. Praesent feugiat sapien aliquet odio. Integer vitae justo. Aliquam
vestibulum fringilla lorem. Sed neque lectus, consectetuer at, consectetuer
sed, eleifend ac, lectus. Nulla facilisi. Pellentesque eget lectus. Proin eu me-
tus. Sed porttitor. In hac habitasse platea dictumst. Suspendisse eu lectus.
Ut mi mi, lacinia sit amet, placerat et, mollis vitae, dui. Sed ante tellus,
tristique ut, iaculis eu, malesuada ac, dui. Mauris nibh leo, facilisis non,
adipiscing quis, ultrices a, dui.

\addcontentsline{toc}{section}{Conclusion}


\section*{Conclusion}
Morbi luctus, wisi viverra faucibus pretium, nibh est placerat odio, nec
commodo wisi enim eget quam. Quisque libero justo, consectetuer a, feugiat
vitae, porttitor eu, libero. Suspendisse sed mauris vitae elit sollicitudin ma-
lesuada. Maecenas ultricies eros sit amet ante. Ut venenatis velit. Maecenas
sed mi eget dui varius euismod. Phasellus aliquet volutpat odio. Vestibulum
ante ipsum primis in faucibus orci luctus et ultrices posuere cubilia Curae ;
Pellentesque sit amet pede ac sem eleifend consectetuer. Nullam elementum,
urna vel imperdiet sodales, elit ipsum pharetra ligula, ac pretium ante justo
a nulla. Curabitur tristique arcu eu metus. Vestibulum lectus. Proin mauris.
Proin eu nunc eu urna hendrerit faucibus. Aliquam auctor, pede consequat
laoreet varius, eros tellus scelerisque quam, pellentesque hendrerit ipsum
dolor sed augue. Nulla nec lacus.

Suspendisse vitae elit. Aliquam arcu neque, ornare in, ullamcorper quis,
commodo eu, libero. Fusce sagittis erat at erat tristique mollis. Maecenas
sapien libero, molestie et, lobortis in, sodales eget, dui. Morbi ultrices rutrum
lorem. Nam elementum ullamcorper leo. Morbi dui. Aliquam sagittis. Nunc
placerat. Pellentesque tristique sodales est. Maecenas imperdiet lacinia velit.
Cras non urna. Morbi eros pede, suscipit ac, varius vel, egestas non, eros.
Praesent malesuada, diam id pretium elementum, eros sem dictum tortor,
vel consectetuer odio sem sed wisiv
%\footnote{Voir \emph{supra} le tableau \nameref{} (tableau \ref{}, page \pageref{}), ainsi que la partie \nameref{}}


\listoftables
\printindex
\printglossary[title=Glossaire]
\tableofcontents
\end{document}
